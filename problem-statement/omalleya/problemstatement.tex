\documentclass[10pt,letterpaper,draftclsnofoot,onecolumn]{IEEEtran}

\usepackage{graphicx}                                        
\usepackage{amssymb}                                         
\usepackage{amsmath}                                         
\usepackage{amsthm}                                          

\usepackage{alltt}                                           
\usepackage{float}
\usepackage{color}
\usepackage{url}

\usepackage{balance}
\usepackage[TABBOTCAP, tight]{subfigure}
\usepackage{enumitem}
%\usepackage{pstricks, pst-node}

\usepackage{geometry}
\geometry{margin=0.75in}

%random comment

\newcommand{\cred}[1]{{\color{red}#1}}
\newcommand{\cblue}[1]{{\color{blue}#1}}

\newcommand{\toc}{\tableofcontents}

\usepackage{hyperref}

\def\name{Aidan O'Malley}

%pull in the necessary preamble matter for pygments output
%\input{pygments.tex}

%% The following metadata will show up in the PDF properties
\hypersetup{
   colorlinks = false,
   urlcolor = black,
   pdfauthor = {\name},
   pdfkeywords = {cs461},
   pdftitle = {CS 461 Problem Statement},
   pdfsubject = {CS 461 Problem Statement},
   pdfpagemode = UseNone
}

\parindent = 0.0 in
\parskip = 0.1 in

\begin{document}

%input the pygmentized output of mt19937ar.c, using a (hopefully) unique name
%this file only exists at compile time. Feel free to change that.

\begin{titlepage}
\title{CS 461 Problem Statement}
\author
{\IEEEauthorblockN{Aidan O'Malley\\
}
\IEEEauthorblockA{
CS 461\\
Fall 2017
}}
    \maketitle
    \vspace{2cm}
    \begin{abstract}
        \noindent This document highlights a problem with the current manner in which music is taught. A solution is then proposed through the creation of an interactive mobile application. Lastly, some metrics that may determine completion of this solution are listed.
    \end{abstract}

\end{titlepage}

\section{Problem}
\noindent The problem our team is trying to solve is that learning music traditionally is much too difficult as it stands today. This is caused by numerous factors which I will begin to explain. One of the biggest reasons why music is so hard to learn is because of the barriers to entry into the world of music theory. 

There is an extremely large number of resources that one could pick from to begin learning music theory and composition techniques, whether it be a teacher, a website, a textbook, or any number of others. In most cases it is very difficult to sift through all of the options to figure out which one of these resources will suit your needs best. Many of these resources involve payment of some sort also. We will need to address these issues with our solution.

Probably the biggest issue we face, is not only are there too many resources, the majority of these resources are providing a lot of unnecessarily difficult information to the user. 

Lukas himself had already finished his undergraduate in music and was in the process of obtaining his master's and even he was still feeling a bit lost. He felt as though he had over time gained a lot of information and general knowledge on music, however, he still couldn't really put the pieces together. That was when a professor mentioned something during one class period which made everything click. Why isn’t there an easier way to begin learning music or just an easier way to reference certain musical patterns to be able to begin creating? This is the core problem that our application intends to address.

\section{Solution}
\noindent There are many aspects that will come together to be able to solve the problem at hand.

A key component of solving this problem will be to reach the largest number of users as possible. A good teacher or book will only ever realistically affect the group of people that are searching them out. We are able to give people the information they need to begin to master music at the palm of their hands. 

We are currently considering multiple cross platform development frameworks so that we will be able to reach the largest audience of mobile users. The two main frameworks we are looking at are Xamarin Studio and React Native. Both of these frameworks would allow us to make an Android and iOS application with essentially the same code base. This will allow us to make one really great product instead of two mediocre ones if we had to split our team's work up between the mobile operating systems. One thing to note would be that using React Native would also make it fairly easy to have a web application to reach even more users. Transferring React Native code to React code is straight forward and would give users another option to access this application.

The other key component of being able to teach music theory is the actual theory that we will be providing. This comes from Lukas’s “Aha” moment years ago and will focus greatly upon the “Circle of Fifths” and the note pattern “BEADGCF”. With these two concepts it is very simple to master a lot of what western music today boils down too. Now that we have defined what we will be teaching, what matters most is how it is presented to the user.

By implementing a simple and interactive interface, we aim to attract all ages and abilities to our application.

\section{Performance Metrics}
\noindent There are a few ways that we can measure our progress as we try to reach our goal solution.

A working application on both the App Store and Google Play Store

The application is easy to understand and the flow is easy to pick up

The application is visually pleasing


\end{document}
